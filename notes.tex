% Autogenerated translation of notes.md by Texpad
% To stop this file being overwritten during the typeset process, please move or remove this header

\documentclass[12pt]{book}
\usepackage{graphicx}
\usepackage{fontspec}
\usepackage[utf8]{inputenc}
\usepackage[a4paper,left=.5in,right=.5in,top=.3in,bottom=0.3in]{geometry}
\setlength\parindent{0pt}
\setlength{\parskip}{\baselineskip}
\setmainfont{Helvetica Neue}
\usepackage{hyperref}
\pagestyle{plain}
\begin{document}

\chapter*{Neuroscience self-study notes}

\hrule
-$>$ Zongru (Doris) Shao $<$-

-$>$ Oct. 2019 $<$-

\section*{resting potential}

\subsection*{parts of a neuron cell}

\begin{itemize}
\item dendrites
\item soma
\item axon
\end{itemize}

\subsection*{resting potential}

\begin{itemize}
\item In real life the actual chloride concentrations inside and outside of a neuron are a little different from that of sodium. Chloride actually has a Nernst potential very close to that of the resting potential (about -70 mV).
\end{itemize}

\end{document}
